\documentclass{article}
\usepackage[backend=biber,style=alphabetic]{biblatex}
\addbibresource{ref.bib}

\usepackage{fontspec}
\setsansfont{Latin Modern Sans}
\renewcommand{\familydefault}{\sfdefault}

\usepackage{graphicx} % Required for inserting images
\usepackage{blindtext}
\usepackage[hidelinks=true]{hyperref}
\usepackage[scale=0.5,text=draft 27/01]{draftwatermark}

\title{Numerical Solutions of PDEs and Applications}
\author{Jack S. Hale and Franck Sueur}
\date{Summer Semester 2024/2025}

\begin{document}

\maketitle
\tableofcontents

\section{Introduction}

This course is an introduction to the numerical solution of partial
differential equations (PDEs). It contains a theoretical part setting the
mathematical foundations necessary for some important numerical methods used to
obtain solutions to some classical PDEs, in particular the finite element
method and the finite difference method. The theoretical part of the course is
supported by the development of a one-dimensional Galerkin finite element code
for the Poisson problem and a one-dimensional finite difference code for a
scalar hyperbolic transport problem.

We largely follow the reference \cite{Q} which is available via the
\href{https://a-z.lu}{National Library's Website} as well as on some less
official websites. In particular, we cover the contents of \cite[Chapters 1, 2,
3, 4, and 14]{Q}.

\section{Syllabus}

\subsection{A brief survey of partial differential equations}

This chapter briefly introduces the notion of linear PDEs and their
classification into elliptic, parabolic and hyperbolic  equations. We mention
some classical examples, mainly issued from physics and engineering, such as
the transport equation, the Laplace equation, the heat equation and the wave
equation. 

\subsection{Elements of functional analysis} 

We introduce the main notions and theoretical results of functional analysis
that are extensively used in the numerical analysis of partial differential
equations. We consider the Riesz representation theorem regarding
representation of continuous linear forms on a Hilbert space, and we survey the
notions of bilinear form, of continuous injection of a Hilbert space into
another, the notion of derivative in the sense of Fréchet, some elements of the
theory of distributions, the basic properties of the Lebesgue and Sobolev
spaces, and the notion of adjoint operator. 

\subsection{Elliptic equations}

We illustrate boundary-value problems for elliptic equations (in one and
several dimensions), present their variational reformulations, treat the
boundary conditions and analyze their well-posedness. Several examples of
physical interest are introduced, in particular the Poisson equation, starting
with the one-dimensional case, for various boundary conditions. We consider
some variational formulations of these problems, and then turn to the
boundary-value problems associated to the Poisson equation in the
two-dimensional case. We establish that under some regularity condition the
weak formulation is equivalent to the strong one. For general elliptic
problems, the Lax-Milgram theorem ensures that the weak formulation is
well-posed. 

\subsection{The Galerkin finite element method for elliptic problems}

We formulate Galerkin’s method for the numerical discretization of elliptic
boundary-value problems and analyze its existence, uniqueness, stability and
convergence features  in an abstract functional setting. We then introduce the
Galerkin finite elements method, first in one dimension, and then in several
dimensions.

\subsection{The Galerkin finite element method -- a numerical example}

This section introduces students to the development of a one-dimensional
Galerkin finite element solver for the Poisson problem using Python. We focus
on building the solver within a provided Jupyter notebook, providing hands-on
experience with the computational and algorithmic aspects of the finite element
method.

\subsection{Finite differences for hyperbolic equations}

The aim of this final chapter is to study classical finite differences
methods for approximating solutions of first-order hyperbolic equations. We
start with exposing some hyperbolic equations starting with the scalar
transport problem in one dimension which we analyze by the method of
characteristics. We also establish an a priori estimate by the energy method.
We then turn to systems of linear hyperbolic equations in one dimension and
give the example of  the wave equation. Then we introduce the finite
difference method, together with its variants: the forward/centered Euler
scheme, the Lax-Friedrichs scheme, the Lax-Wendroff scheme, in the case of the
scalar transport problem, before to move to more general cases, for which we
analyze the consistency, stability, convergence, dissipation and dispersion
properties of the finite difference methods 

\subsection{Finite differences -- a numerical example}

This section introduces students to the development of a one-dimensional finite
difference solver for a hyperbolic transport problem using Python. We focus on
building the solver within a provided Jupyter notebook, providing hands-on
experience with the computational and algorithmic aspects of the finite
difference method.

\section{Practical matters}

\subsection{Assumed knowledge}

It is assumed that students have taken \emph{Functional Analysis} and
\emph{Numerical Analysis} courses in MAMATH, or have equivalent knowledge.

\subsection{Dropping the course}

Students cannot drop this course after the two week trial period at the start
of the semester.

\subsection{Teaching}

For \emph{practical session} must bring a laptop with Python version 3.10 or
later installed (recommended), or with access to
\href{https://colab.research.google.com}{Google Colaboratory}.

\subsection{Assessment}

Assessment is via coursework (30\%) and final examination (70\%). The due date
for the coursework will be set at a later time.

\subsection{Attendance policy}

Attendance at the \emph{lectures} is strongly advised.

Attendance at the \emph{practical sessions} is mandatory as they relate to the
coursework - one absence will be excused. Further absences must be supported by
documentation according to the study regulations and a meeting with the study
programme director and/or instructors.

\subsection{Retake policy}

Failure on the coursework can be compensated through the examination, and vice
versa, as long as the final mark for the course is greater than or equal to 10.

In the event that the final mark is less than 10:

\begin{itemize}
\item The student may request a retake the examination once.
\item The student may not resubmit the coursework.
\end{itemize}

If the final mark remains less than 10 the student has failed, and can take the
entire course again in the next semester that it is offered.

\subsection{Communication}

Electronic communication to students as a group will be via the University's
Moodle. Private communications from students to the instructors must be made
via the University email.

\printbibliography

\end{document}
