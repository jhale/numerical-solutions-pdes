\documentclass{article}
\usepackage[backend=biber,style=alphabetic]{biblatex}
\addbibresource{ref.bib}

\usepackage{fontspec}
\setsansfont{Latin Modern Sans}
\renewcommand{\familydefault}{\sfdefault}

\usepackage{graphicx} % Required for inserting images
\usepackage{blindtext}
\usepackage{hyperref}

\title{Numerical Solutions of PDEs and Applications}
\author{Jack S. Hale and Franck Sueur}
\date{Summer Semester 2025}

\begin{document}

\maketitle

\section{Introduction}

This course is an introduction to the numerical solution of partial
differential equations (PDEs). It contains a theoretical part setting the
mathematical foundations necessary for some important numerical methods used to
obtain solutions to some classical PDEs, in particular the finite element
method and the finite difference method. The theoretical part of the course is
supported by the development of a one-dimensional Galerkin finite element code
for the Poisson problem and a one-dimensional finite difference code for a
scalar hyperbolic transport problem.

We largely follow the reference \cite{Q} which is available via the
\href{https://a-z.lu}{National Library's Website} as well as on some less
official website. In particular, we cover the contents of \cite[Chapters 1, 2,
3, 4, and 14]{Q}.

\section{Organisation}

\subsection{Teaching}

For \emph{practical session} must bring a laptop with Python version 3.10 or
later installed (recommended), or with access to Google Colaboratory.

\subsection{Assessment}

\section{Syllabus}

\subsection{A brief survey of partial differential equations}

This chapter briefly introduces the notion of linear PDEs and their
classification into elliptic, parabolic and hyperbolic  equations. We mention
some classical examples, mainly issued from physics and engineering, such as
the transport equation, the Laplace equation, the heat equation and the wave
equation. 

\subsection{Elements of functional analysis} 

We introduce the main notions and theoretical results of functional analysis
that are extensively used throughout the course. We consider the Riesz
representation theorem regarding representation of continuous linear forms on a
Hilbert space, and we survey the notions of bilinear form, of continuous
injection of a Hilbert space into another, the notion of derivative in the
sense of Fréchet, some elements of the theory of distributions, the basic
properties of the Lebesgue and Sobolev spaces, and the notion of adjoint
operator. 

\subsection{Elliptic equations}

We illustrate boundary-value problems for elliptic equations (in one and
several dimensions), present their variational reformulations, treat the
boundary conditions and analyze their well-posedness. Several examples of
physical interest are introduced, in particular the Poisson equation, starting
with the one-dimensional case, for various boundary conditions. We consider
some variational formulations of these problems, and then turn to the
boundary-value problems associated to the Poisson equation in the
two-dimensional case. We establish that under some regularity condition the
weak formulation is equivalent to the strong one. For general elliptic
problems, the Lax-Milgram theorem  ensures that the weak formulation is
well-posed. 

\subsection{The Galerkin finite element method for elliptic problems}

We formulate Galerkin’s method for the numerical discretization of elliptic
boundary-value problems and analyze its existence, uniqueness, stability and
convergence features  in an abstract functional setting. We then introduce the
Galerkin finite elements method, first in one dimension,  and then in several
dimensions.

\subsection{Numerical example}

\subsection{Finite differences for hyperbolic equations}

The aim of this final chapter is to address classical finite differences
methods to  approximate solutions of first-order hyperbolic equations. We
start with exposing some hyperbolic equations starting with the scalar
transport problem in one dimension which we analyze by the method of
characteristics. We also establish an a priori estimate by the energy method.
We then turn to systems of linear hyperbolic equations in one dimension and
give the example of  the wave equation. Then we introduce the finite
difference method, together with its variants: the forward/centered Euler
scheme, the Lax-Friedrichs scheme, the Lax-Wendroff scheme, in the case of the
scalar transport problem, before to move to more general cases, for which we
analyze the consistency, stability, convergence, dissipation and dispersion
properties of the finite difference methods 

\subsection{Numerical example}


\printbibliography

\end{document}
